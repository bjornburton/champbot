\input cwebmac

% piruett

\nocon % omit table of contents
\datethis % print date on listing


\N{1}{1}Introduction. This is the firmware portion of the propulsion system for
our
Champbot.
It features separate thrust and steering as well as piruett turning.

This will facilitate motion by taking ``thrust'' and ``radius'' pulse-width
inputs from the Futaba RC receiver by converting them to the appropriate motor
actions.
These are from Channel 2 at analog input A1 and channel 1 at A0, respectivily.
The action will be similar to driving an RC car or boat.
By keeping it natural, it should be easier to navigate the course than with a
skid-steer style control.

\fi

\N{1}{2}Implementation.
The Futaba receiver has two PWC channels.
The pulse-width from the receiver is at 20 ms intervals.
The on-time ranges from 1000--2000 ms, including trim.
1500~ ms is the pulse-width for stopped.
The levers cover $\pm$0.4~ms and the trim covers the balance.

Both pulse-width inputs will need some dead-band to allow for full stop.

Port motor pulse will be applied to ???, starboard will be at ???.
The median time will be subtracted from them for a pair of signed values
thrust and radius. The value will be scaled.

The thrust and radius will be translated to power to the
port and starboard motors. When near median the motors will be disabled.
The motors will also be disabled when there are no input pulses.
Each motor need direction and power so that's 4 signals of output.

The radius control will also be the rotate control, if thrust is zero.

Adding the two signals of input, I need more I/O than the original trinket has.
So---I now have moved to a \$10 Pro Trinket with far more capability.
It has an ATmega328, just like and Arduino.

The ATmega328 has a fancy 16 bit PWM with two comparators, Timer 1.
This has an ``Input Capture Unit'' that may be used for PWC decoding.
PWC being the type of signal from the RC receiver.
That seems like as elegant a solutioni as I will find and it is recommended by
Atmel to use it for this purpose.

The best way to use this nice feature is to
take the PWC signals into the MUX, through the comparator and into the Input
Capture Unit.

One of the other timers will do more than fine for the two motors.

For the PWC measurement, this app note, AVR135, is helpful:
www.atmel.com/images/doc8014.pdf

In the datasheet, this section is helpful: 16.6.3

An interesting thing about this Futaba receiver is that the pulses are in
series.
The channel two's pulse is first, followed the channel one.
In fact, channel two's fall is perfectly aligned with channel one's rise.
This means that it will be possible to capture all of the pulses.

After the two pulses are captured, there's an 18~ms dead-time before the next
round. That's over 250,000 clock cycles.
This will provide ample time to do math and set the motor PWMs.


Extensive use was made of the datasheet, Atmel
``Atmel-8271I-AVR- ATmega-Datasheet\_10/2014''.

\Y\B\X6:Include\X\6
\X7:Types\X\6
\X9:Prototypes\X\6
\X10:Global variables\X\par
\fi

\M{3}\PB{\.{"F\_CPU"}} is used to convey the Trinket Pro clock rate.
\Y\B\4\D$\.{F\_CPU}$ \5
\T{16000000\$U\$L}\par
\B\4\D$\.{BAUD}$ \5
\T{9600}\par
\fi

\M{4}Here are some Boolean definitions that are used.
\Y\B\4\D$\.{ON}$ \5
\T{1}\par
\B\4\D$\.{OFF}$ \5
\T{0}\par
\B\4\D$\.{SET}$ \5
\T{1}\par
\B\4\D$\.{CLEAR}$ \5
\T{0}\par
\B\4\D$\.{TRUE}$ \5
\T{1}\par
\B\4\D$\.{FALSE}$ \5
\T{0}\par
\B\4\D$\.{FORWARD}$ \5
\T{1}\par
\B\4\D$\.{REVERSE}$ \5
\T{0}\par
\B\4\D$\.{CLOSED}$ \5
\T{1}\par
\B\4\D$\.{OPEN}$ \5
\T{0}\par
\fi

\M{5}Here are some other definitions.
\Y\B\4\D$\.{CH2RISE}$ \5
\T{0}\par
\B\4\D$\.{CH2FALL}$ \5
\T{1}\par
\B\4\D$\.{CH1FALL}$ \5
\T{2}\par
\B\4\D$\.{MAX\_DUTYCYCLE}$ \5
\T{98}\SHC{ 98\% to support charge pump of bridge-driver }\par
\fi

\M{6}\B\X6:Include\X${}\E{}$\6
\8\#\&{include} \.{<avr/io.h>}\SHC{ need some port access }\6
\8\#\&{include} \.{<util/delay.h>}\SHC{ need to delay }\6
\8\#\&{include} \.{<avr/interrupt.h>}\SHC{ have need of an interrupt }\6
\8\#\&{include} \.{<avr/sleep.h>}\SHC{ have need of sleep }\6
\8\#\&{include} \.{<avr/wdt.h>}\SHC{ have need of watchdog }\6
\8\#\&{include} \.{<stdlib.h>}\6
\8\#\&{include} \.{<stdint.h>}\par
\U2.\fi

\M{7}Here is a structure type to keep track of the state of remote-control
input, e.g. servo timing.

\Y\B\4\X7:Types\X${}\E{}$\6
\&{typedef} \&{struct} ${}\{{}$\1\6
\\{uint16\_t}\\{ch2rise};\SHC{ pwc edge }\6
\\{uint16\_t}\\{ch2fall};\SHC{ pwc edge }\6
\\{uint16\_t}\\{ch1fall};\SHC{ pwc edge }\6
\\{uint16\_t}\\{ch1duration};\6
\\{uint16\_t}\\{ch2duration};\6
\\{uint16\_t}\\{minIn};\SHC{ input, minimum }\6
\\{uint16\_t}\\{maxIn};\SHC{ input, maximum }\6
\\{uint8\_t}\\{edge};\6
\\{uint8\_t}\\{lostSignal};\2\6
${}\}{}$ \&{inputStruct};\par
\A8.
\U2.\fi

\M{8}Here is a structure type to keep track of the state of translation items.
\Y\B\4\X7:Types\X${}\mathrel+\E{}$\6
\&{typedef} \&{struct} ${}\{{}$\1\6
\\{int16\_t}\\{thrust};\SHC{ -255 to 255 }\6
\\{int16\_t}\\{radius};\SHC{ -255 to 255 }\6
\\{int16\_t}\\{track};\SHC{    1 to 255 }\6
\\{int16\_t}\\{starboardOut};\SHC{ -255 to 255 }\6
\\{int16\_t}\\{portOut};\SHC{ -255 to 255 }\6
\\{int16\_t}\\{minOut};\SHC{ output, minimum }\6
\\{int16\_t}\\{maxOut};\SHC{ output, maximum }\6
\\{int8\_t}\\{deadBand};\SHC{ width of zero in terms of output units }\2\6
${}\}{}$ \&{transStruct};\par
\fi

\M{9}\B\X9:Prototypes\X${}\E{}$\6
\&{void} \\{relayCntl}(\\{int8\_t}\\{state});\6
\&{void} \\{ledCntl}(\\{int8\_t}\\{state});\6
\&{void} \\{portDirection}(\\{int8\_t}\\{state});\6
\&{void} \\{starboardDirection}(\\{int8\_t}\\{state});\6
\&{void} \\{pwcCalc}(\&{inputStruct} ${}{*});{}$\6
\&{void} \\{edgeSelect}(\&{inputStruct} ${}{*});{}$\7
\\{uint16\_t}\\{scaler}(\&{inputStruct} ${}{*},\39{}$\&{transStruct} ${}{*},\39%
\\{uint16\_t}\\{input});{}$\7
\&{void} \\{translate}(\&{transStruct} ${}{*});{}$\6
\&{void} \\{setPwm}(\&{transStruct} ${}{*});{}$\6
\&{void} \\{lostSignal}(\&{inputStruct} ${}{*}){}$;\par
\U2.\fi

\M{10}
My lone global variable is a function pointer.
This lets me pass arguments to the actual interrupt handlers.
This pointer gets the appropriate function attached by the \PB{\.{"ISR()"}}
function.

The input structure is to contain all of the external inputs.

\Y\B\4\X10:Global variables\X${}\E{}$\6
$\&{void}({*}\\{handleIrq}){}$(\&{inputStruct} ${}{*})\K\NULL{}$;\par
\U2.\fi

\M{11}
Here is \PB{\\{main}(\,)}.
\Y\B\&{int} \\{main}(\&{void})\1\1 $\{{}$\Y\par
\fi

\M{12}
The Futaba receiver leads with channel two, rising edge, so we will start
looking for that by setting \PB{\.{"edge"}} to look for a rise on channel 2.

Until we have collected the edges we will assume there is no signal.
\Y\B\&{inputStruct} \\{input\_s} $\K$ $\{$ $.$ $\\{edge}\K\.{CH2RISE}$ $,$ $.$
$\\{minIn}\K\T{14970}$ $,{}$\SHC{ ticks for hard right or down }\6
$.$ $\\{maxIn}\K\T{27530}$ $,{}$\SHC{ ticks for hard left or up }\6
$.$ $\\{lostSignal}\K{}$\.{TRUE}\SHC{ we need to wait for edges before we know
}\6
$\}$  ;\par
\fi

\M{13}
Center position of the controller results in a count of  about 21250,
hard left, or up, with trim reports about 29100 and hard right, or down,
with trim reports about 13400.

About 4/5ths of that range are the full swing of the stick, without trim.
This is from about 14970 and 27530 ticks.

This \PB{\.{"inputScale\_s"}} structure holds the parameters used in the scaler
function.
The \PB{\.{"In"}} numbers are raw from the Input Capture Register.

At some point a calibration feature could be added which could populate these
but the numbers here were from trial and error and seem good.
\Y\B\&{transStruct} \\{translation\_s} $\K$ $\{$ $.$ $\\{minOut}\K{-}\T{255}$
$,$ $.$ $\\{maxOut}\K\T{255}$ $,$ $.$ $\\{deadBand}\K\T{10}$ $\}$  ;\6
\\{cli}(\,);\SHC{disable interrupts during setup }\7
\X37:Initialize the inputs and capture mode\X\X34:Initialize pin outputs\X%
\\{cli}(\,);\SHC{ disable interrupts }\6
\X39:Initialize watchdog timer\X\Y\par
\fi

\M{14}
Of course, any interrupt function requires that bit ``Global Interrupt Enable''
is set; usually done through calling \PB{\.{"sei()"}}.
\Y\B\\{sei}(\,);\Y\par
\fi

\M{15}

The PWM is used to control port and starboard motors through OC0A (D5) and
OC0B (D6), respectivly.
\Y\B\X41:Initialize the Timer Counter 0 for PWM\X\par
\fi

\M{16}
Rather than burning loops, waiting the ballance of 18~ms for something to
happen, the \PB{\.{"sleep"}} mode is used.
The specific type of sleep is \PB{\.{"idle"}}.
In idle, execution stops but timers continue.
Interrupts are used to wake it up.

It's important to note that an ISR procedure must be defined to allow the
program to step past the sleep statement.
\Y\B\X35:Configure to idle on sleep\X\7
\\{ledCntl}(\.{OFF});\par
\fi

\M{17}
Calling edgeSelect will get it ready for the first rising edge of channel 2.
\Y\B$\\{edgeSelect}({\AND}\\{input\_s}){}$;\par
\fi

\M{18}
This is the loop that does the work.
It should spend most of its time in ``sleep\_mode'', comming out at each
interrupt event caused by an edge.

\Y\B\&{for} ( ;  ; \,) $\{{}$\Y\par
\fi

\M{19}
Now that a loop is started, the PWM is given an initial value and we wait in
\PB{\.{"idle"}} for the edge on the channel selected. Each sucessive loop will
finish
in the same way.
\Y\B$\\{setPwm}({\AND}\\{translation\_s});{}$\6
\\{sleep\_mode}(\,);\SHC{ idle }\par
\fi

\M{20}
If execution arrives here, some interrupt has woken it from sleep and some
vector has possibly run.
The pointer \PB{\.{"handleIrq"}} will be assigned the value of the responsible
function.
\Y\B\&{if} ${}(\\{handleIrq}\I\NULL{}$)\SHC{ in case it woke for some other
reason }\6
${}\{{}$\1\7
${}\\{handleIrq}({\AND}\\{input\_s});{}$\6
${}\\{handleIrq}\K\NULL{}$;\SHC{ reset so that the action cannot be repeated }\6
\4${}\}{}$\SHC{ end if handleirq }\2\6
${}\\{translation\_s}.\\{radius}\K\\{scaler}({\AND}\\{input\_s},\39{\AND}%
\\{translation\_s},\39\\{input\_s}.\\{ch1duration});{}$\6
${}\\{translation\_s}.\\{thrust}\K\\{scaler}({\AND}\\{input\_s},\39{\AND}%
\\{translation\_s},\39\\{input\_s}.\\{ch2duration});{}$\6
${}\\{translation\_s}.\\{track}\K\T{100}{}$;\SHC{ represent unit-less
prop-to-prop distance }\6
${}\\{translate}({\AND}\\{translation\_s}){}$;\par
\fi

\M{21}
Some temporary test code here.
\Y\B\&{if} ${}(\\{translation\_s}.\\{portOut}\E\T{0}){}$\1\5
\\{ledCntl}(\.{ON});\2\6
\&{else}\1\5
\\{ledCntl}(\.{OFF});\2\7
$\}{}$\SHC{ end for }\7
\&{return} \T{0};\SHC{ it's the right thing to do! }\7
$\}{}$\SHC{ end main() }\par
\fi

\M{22}
Here is the ISR that fires at each captured edge.
\Y\B\.{ISR}(\\{TIMER1\_CAPT\_vect})\1\1\2\2\6
${}\{{}$\1\7
${}\\{handleIrq}\K{\AND}\\{pwcCalc};{}$\6
\4${}\}{}$\2\7
\.{ISR}(\\{WDT\_vect})\1\1\2\2\6
${}\{{}$\1\7
${}\\{handleIrq}\K{\AND}\\{lostSignal};{}$\6
\4${}\}{}$\2\par
\fi

\M{23}
This procedure computes the durations from the PWC signal edge capture values
from the Input Capture Unit.
With the levers centered the durations should be about 1.5~ms so at 16~ Mhz
the count should be near 24000.
The range should be 17600 to 30400 for 12800 counts, well within the range
of the 64 kib of the 16 bit register..


\Y\B\&{void} \\{pwcCalc}(\&{inputStruct} ${}{*}\\{input\_s}){}$\1\1 $\{{}$\Y\par
\fi

\M{24}
On the falling edges we can compute the durations using modulus subtraction
and then set the edge index for the next edge.
Channel 2 leads so that rise is first.

Arrival at the last case establishes that there was a signal and clears
the flag and resets the watchdog timer.
\Y\B\&{switch} ${}(\\{input\_s}\MG\\{edge}){}$\5
${}\{{}$\1\6
\4\&{case} \.{CH2RISE}:\5
${}\\{input\_s}\MG\\{ch2rise}\K\.{ICR1};{}$\6
${}\\{input\_s}\MG\\{edge}\K\.{CH2FALL};{}$\6
\&{break};\6
\4\&{case} \.{CH2FALL}:\5
${}\\{input\_s}\MG\\{ch2fall}\K\.{ICR1};{}$\6
${}\\{input\_s}\MG\\{ch2duration}\K\\{input\_s}\MG\\{ch2fall}-\\{input\_s}\MG%
\\{ch2rise};{}$\6
${}\\{input\_s}\MG\\{edge}\K\.{CH1FALL};{}$\6
\&{break};\6
\4\&{case} \.{CH1FALL}:\5
${}\\{input\_s}\MG\\{ch1fall}\K\.{ICR1};{}$\6
${}\\{input\_s}\MG\\{ch1duration}\K\\{input\_s}\MG\\{ch1fall}-\\{input\_s}\MG%
\\{ch2fall};{}$\6
${}\\{input\_s}\MG\\{edge}\K\.{CH2RISE};{}$\6
${}\\{input\_s}\MG\\{lostSignal}\K\.{FALSE}{}$;\SHC{ signal seems OK now }\6
\\{wdt\_reset}(\,);\SHC{ watchdog timer is reset at each edge capture }\6
\4${}\}{}$\2\6
\\{edgeSelect}(\\{input\_s});\7
$\}{}$\par
\fi

\M{25}
This procedure sets output to zero in the event of a lost signal.
\Y\B\&{void} \\{lostSignal}(\&{inputStruct} ${}{*}\\{input\_s}){}$\1\1\2\2\6
${}\{{}$\1\7
${}\\{input\_s}\MG\\{lostSignal}\K\.{TRUE};{}$\6
${}\\{input\_s}\MG\\{edge}\K\.{CH2RISE}{}$;\SHC{ Back to first step }\6
\\{edgeSelect}(\\{input\_s});\6
\4${}\}{}$\2\par
\fi

\M{26}

The procedure edgeSelect configures the Input Capture unit to capture on the
expected edge type.

\Y\B\&{void} \\{edgeSelect}(\&{inputStruct} ${}{*}\\{input\_s}){}$\1\1 $\{{}$\7
\&{switch} ${}(\\{input\_s}\MG\\{edge}){}$\5
${}\{{}$\1\6
\4\&{case} \.{CH2RISE}:\SHC{ wait for rising edge on servo channel 2 }\6
${}\.{ADMUX}\MRL{{\OR}{\K}}(\T{1}\LL\.{MUX0}){}$;\SHC{ Set to mux channel 1 }\6
${}\.{TCCR1B}\MRL{{\OR}{\K}}(\T{1}\LL\.{ICES1}){}$;\SHC{ Rising edge (23.3.2) }%
\6
\&{break};\6
\4\&{case} \.{CH2FALL}:\5
${}\.{ADMUX}\MRL{{\OR}{\K}}(\T{1}\LL\.{MUX0}){}$;\SHC{ Set to mux channel 1 }\6
${}\.{TCCR1B}\MRL{\AND{\K}}\CM(\T{1}\LL\.{ICES1}){}$;\SHC{ Falling edge
(23.3.2) }\6
\&{break};\6
\4\&{case} \.{CH1FALL}:\5
${}\.{ADMUX}\MRL{\AND{\K}}\CM(\T{1}\LL\.{MUX0}){}$;\SHC{ Set to mux channel 0 }%
\6
${}\.{TCCR1B}\MRL{\AND{\K}}\CM(\T{1}\LL\.{ICES1}){}$;\SHC{ Falling edge
(23.3.2) }\6
\4${}\}{}$\2\par
\fi

\M{27}
Since the edge has been changed, the Input Capture Flag should be cleared.
It's odd but clearing it involves writing a one to it.
\Y\B$\.{TIFR1}\MRL{{\OR}{\K}}(\T{1}\LL\.{ICF1}){}$;\SHC{ (per 16.6.3) }\7
$\}{}$\par
\fi

\M{28}
Here is a simple procedure to flip the LED on or off.
\Y\B\&{void} \\{ledCntl}(\\{int8\_t}\\{state})\1\1\2\2\6
${}\{{}$\1\6
${}\.{PORTB}\K\\{state}\?\.{PORTB}\OR(\T{1}\LL\.{PORTB5}):\.{PORTB}\AND\CM(%
\T{1}\LL\.{PORTB5});{}$\6
\4${}\}{}$\2\par
\fi

\M{29}
Here is a simple procedure to flip the Relay Closed or Open from pin \#8.
\Y\B\&{void} \\{relayCntl}(\\{int8\_t}\\{state})\1\1\2\2\6
${}\{{}$\1\6
${}\.{PORTB}\K\\{state}\?\.{PORTB}\OR(\T{1}\LL\.{PORTB0}):\.{PORTB}\AND\CM(%
\T{1}\LL\.{PORTB0});{}$\6
\4${}\}{}$\2\par
\fi

\M{30}
Here is a simple procedure to reverse thrust on the port motor.
\Y\B\&{void} \\{portDirection}(\\{int8\_t}\\{state})\1\1\2\2\6
${}\{{}$\1\6
${}\.{PORTD}\K\\{state}\?\.{PORTD}\OR(\T{1}\LL\.{PORTD3}):\.{PORTD}\AND\CM(%
\T{1}\LL\.{PORTD3});{}$\6
\4${}\}{}$\2\par
\fi

\M{31}
Here is a simple procedure to reverse thrust on the starboard motor.
\Y\B\&{void} \\{starboardDirection}(\\{int8\_t}\\{state})\1\1\2\2\6
${}\{{}$\1\6
${}\.{PORTD}\K\\{state}\?\.{PORTD}\OR(\T{1}\LL\.{PORTD4}):\.{PORTD}\AND\CM(%
\T{1}\LL\.{PORTD4});{}$\6
\4${}\}{}$\2\par
\fi

\M{32}

\fi

\N{1}{33}Supporting routines, functions, procedures and configuration
blocks.

\fi

\M{34}\B\X34:Initialize pin outputs\X${}\E{}$\SHC{ set the led port direction;
This is pin \#17 }\6
$\.{DDRB}\MRL{{\OR}{\K}}(\T{1}\LL\.{DDB5}){}$;\SHC{ set the relay port
direction; This is pin \#8 }\6
${}\.{DDRB}\MRL{{\OR}{\K}}(\T{1}\LL\.{DDB0}){}$;\SHC{ 14.4.9 DDRD – The Port
D Data Direction Register }\SHC{ port and starboard pwm outputs }\6
${}\.{DDRD}\MRL{{\OR}{\K}}((\T{1}\LL\.{DDD5})\OR(\T{1}\LL\.{DDD6})){}$;\SHC{
Data direction to output (sec 14.3.3) }\SHC{ port and starboard direction
outputs }\6
${}\.{DDRD}\MRL{{\OR}{\K}}((\T{1}\LL\.{DDD3})\OR(\T{1}\LL\.{DDD4})){}$;\SHC{
Data direction to output (sec 14.3.3) }\par
\U13.\fi

\M{35}\B\X35:Configure to idle on sleep\X${}\E{}$\6
${}\{{}$\1\6
${}\.{SMCR}\MRL{\AND{\K}}\CM((\T{1}\LL\.{SM2})\OR(\T{1}\LL\.{SM1})\OR(\T{1}\LL%
\.{SM0}));{}$\6
\4${}\}{}$\2\par
\U16.\fi

\M{36}
To enable this interrupt, set the ACIE bit of register ACSR.
\fi

\M{37}\B\X37:Initialize the inputs and capture mode\X${}\E{}$\6
${}\{{}$\SHC{ ADCSRA – ADC Control and Status Register A }\1\6
${}\.{ADCSRA}\MRL{\AND{\K}}\CM(\T{1}\LL\.{ADEN}){}$;\SHC{ Conn the MUX to (-)
input of comparator (sec 23.2) }\SHC{ 23.3.1 ADCSRB – ADC Control and Status
Register B }\6
${}\.{ADCSRB}\MRL{{\OR}{\K}}(\T{1}\LL\.{ACME}){}$;\SHC{ Conn the MUX to (-)
input of comparator (sec 23.2) }\SHC{ 24.9.5 DIDR0 – Digital Input Disable
Register 0 }\6
${}\.{DIDR0}\MRL{{\OR}{\K}}((\T{1}\LL\.{AIN1D})\OR(\T{1}\LL\.{AIN0D})){}$;\SHC{
Disable digital inputs (sec 24.9.5) }\SHC{ 23.3.2 ACSR – Analog Comparator
Control and Status Register }\6
${}\.{ACSR}\MRL{{\OR}{\K}}(\T{1}\LL\.{ACBG}){}$;\SHC{ Connect + input to the
band-gap ref (sec 23.3.2) }\6
${}\.{ACSR}\MRL{{\OR}{\K}}(\T{1}\LL\.{ACIC}){}$;\SHC{ Enable input capture mode
(sec 23.3.2) }\6
${}\.{ACSR}\MRL{{\OR}{\K}}(\T{1}\LL\.{ACIS1}){}$;\SHC{ Set for both rising and
falling edge (sec 23.3.2) }\SHC{ 16.11.8 TIMSK1 – Timer/Counter1 Interrupt
Mask Register }\6
${}\.{TIMSK1}\MRL{{\OR}{\K}}(\T{1}\LL\.{ICIE1}){}$;\SHC{ Enable input capture
interrupt (sec 16.11.8) }\SHC{ 16.11.2 TCCR1B – Timer/Counter1 Control
Register B }\6
${}\.{TCCR1B}\MRL{{\OR}{\K}}(\T{1}\LL\.{ICNC1}){}$;\SHC{ Enable input capture
noise canceling (sec 16.11.2) }\6
${}\.{TCCR1B}\MRL{{\OR}{\K}}(\T{1}\LL\.{CS10}){}$;\SHC{ No Prescale. Just count
the main clock (sec 16.11.2) }\SHC{ 24.9.1 ADMUX – ADC Multiplexer Selection
Register }\6
${}\.{ADMUX}\MRL{\AND{\K}}\CM((\T{1}\LL\.{MUX2})\OR(\T{1}\LL\.{MUX1})\OR(\T{1}%
\LL\.{MUX0})){}$;\SHC{ Set to mux channel 0 }\6
\4${}\}{}$\2\par
\U13.\fi

\M{38}
See section 11.8 in the datasheet for details on the Watchdog Timer.
This is in the ``Interrupt Mode''.
\fi

\M{39}\B\X39:Initialize watchdog timer\X${}\E{}$\6
${}\{{}$\1\6
${}\.{WDTCSR}\MRL{{\OR}{\K}}(\T{1}\LL\.{WDCE})\OR(\T{1}\LL\.{WDE});{}$\6
${}\.{WDTCSR}\K(\T{1}\LL\.{WDIE})\OR(\T{1}\LL\.{WDP2}){}$;\SHC{ reset after
about 0.25 seconds }\6
\4${}\}{}$\2\par
\U13.\fi

\M{40}
PWM setup isn't too scary.
Timer Count 0 is configured for ``Phase Correct'' PWM which, according to the
datasheet, is preferred for motor control.
OC0A (port) and OC0B (starboard) are set to clear on a match which creates a
non-inverting PWM.
The prescaler is set to clk/8 and with a 16 MHz clock the $f$ is about 3922 Hz.
\fi

\M{41}\B\X41:Initialize the Timer Counter 0 for PWM\X${}\E{}$\6
${}\{{}$\SHC{ 15.9.1 TCCR0A – Timer/Counter Control Register A }\1\6
${}\.{TCCR0A}\MRL{{\OR}{\K}}(\T{1}\LL\.{WGM00}){}$;\SHC{ Phase correct mode of
PWM (table 15-9) }\6
${}\.{TCCR0A}\MRL{{\OR}{\K}}(\T{1}\LL\.{COM0A1}){}$;\SHC{ Clear on Comparator A
match (table 15-4) }\6
${}\.{TCCR0A}\MRL{{\OR}{\K}}(\T{1}\LL\.{COM0B1}){}$;\SHC{ Clear on Comparator B
match (table 15-7) }\SHC{ 15.9.2 TCCR0B – Timer/Counter Control Register B }\6
${}\.{TCCR0B}\MRL{{\OR}{\K}}(\T{1}\LL\.{CS01}){}$;\SHC{ Prescaler set to clk/8
(table 15-9) }\6
\4${}\}{}$\2\par
\U15.\fi

\M{42}
The scaler function takes an input, as in times from the Input Capture
Register and returns a value scaled by the parameters in structure
\PB{\.{"inputScale\_s"}}.
\Y\B\\{uint16\_t}\\{scaler}(\&{inputStruct} ${}{*}\\{input\_s},\39{}$%
\&{transStruct} ${}{*}\\{trans\_s},\39\\{uint16\_t}\\{input}){}$\1\1 $\{{}$\7
\\{uint16\_t}\\{solution};\par
\fi

\M{43}
First, we can solve for the obvious cases.
One is where there is no signal.
The other is where the input exceeds the range.
This can easily happen if the trim is shifted.
\Y\B\&{if} ${}(\\{input\_s}\MG\\{lostSignal}\E\.{TRUE}{}$)\SHC{ no valid signal
}\1\6
\&{return} \T{0};\2\6
\&{if} ${}(\\{input}>\\{input\_s}\MG\\{maxIn}){}$\1\5
\&{return} \\{trans\_s}${}\MG\\{maxOut};{}$\2\6
\&{if} ${}(\\{input}<\\{input\_s}\MG\\{minIn}){}$\1\5
\&{return} \\{trans\_s}${}\MG\\{minOut}{}$;\2\par
\fi

\M{44}
If it's not that simple, then compute the gain and offset and then continue in
the usual way.
This is not really an effecient method, recomputing gain and offset every time
but we are not in a rush and it makes it easier since, if something changes,
I don't have to manualy compute and enter these values, also the code is all in
one place.

The constant \PB{\.{"ampFact"}} amplifies it so I can take advantage of the
extra
bits for precision.

Dead-band is applied when it returns.
\Y\B\&{const} \\{int32\_t}\\{ampFact}${}\K\T{128\$L}{}$;\SHC{ factor for
precision }\7
${}\\{int32\_t}\\{gain}\K(\\{ampFact}*(\\{int32\_t})(\\{input\_s}\MG\\{maxIn}-%
\\{input\_s}\MG\\{minIn}))/(\\{int32\_t})(\\{trans\_s}\MG\\{maxOut}-\\{trans%
\_s}\MG\\{minOut});{}$\6
${}\\{int32\_t}\\{offset}\K((\\{ampFact}*(\\{int32\_t})\\{input\_s}\MG%
\\{minIn})/\\{gain})-(\\{int32\_t})\\{trans\_s}\MG\\{minOut};{}$\6
${}\\{solution}\K(\\{ampFact}*(\\{int32\_t})\\{input}/\\{gain})-\\{offset};{}$\6
\&{return} ${}(\\{solution}>\\{trans\_s}\MG\\{deadBand})\?\\{solution}:\T{0};$
$\}{}$\par
\fi

\M{45}
We need a way to translate \PB{\.{"thrust"}} and \PB{\.{"radius"}} in order to
carve a
\PB{\.{"turn"}}. This procedure should do this but it's not going to be perfect
as
drag and slippage make thrust increase progressivly more than speed.
It should steer OK as long as the speed is constant and small changes in speed
should not be too disruptive.

This procedure is intended for values from -255 to 255.
\Y\B\&{void} \\{translate}(\&{transStruct} ${}{*}\\{trans\_s}){}$\1\1 $\{$ $%
\\{int16\_t}\\{speed}\K\\{trans\_s}\MG\\{thrust}{}$;\SHC{ this is not true,
just assuming }\6
\\{int16\_t}\\{rotation};\6
\\{int16\_t}\\{difference};\7
\&{const} \\{int16\_t}\\{max}${}\K(\.{MAX\_DUTYCYCLE}*\.{UINT8\_MAX})/%
\T{100};{}$\6
\&{const} \\{int16\_t}\\{ampFact}${}\K\T{128}{}$;\SHC{ factor for precision }%
\par
\fi

\M{46}
Here we convert desired radius to thrust-difference by scaling to speed.
Then that difference is converted to rotation by scaling it with \PB{%
\.{"track"}}.
The radius sensitivity is adjusted by changing the value of \PB{\.{"track"}}.
\Y\B$\\{difference}\K(\\{speed}*((\\{ampFact}*\\{trans\_s}\MG\\{radius})/%
\.{UINT8\_MAX}))/\\{ampFact};{}$\6
${}\\{rotation}\K(\\{trans\_s}\MG\\{track}*((\\{ampFact}*\\{difference})/%
\.{UINT8\_MAX}))/\\{ampFact}{}$;\par
\fi

\M{47}
Any rotation involves one motor turning faster than the other.
At some point, faster is not possible and so the requiered clipping is here.

\PB{\.{"max"}} is set at to support the limit of the bridge-driver's
charge-pump.
\Y\B\&{if} ${}((\\{speed}-\\{rotation})\G\\{max}){}$\1\5
${}\\{trans\_s}\MG\\{portOut}\K\\{max};{}$\2\6
\&{else} \&{if} ${}((\\{speed}-\\{rotation})\Z{-}\\{max}){}$\1\5
${}\\{trans\_s}\MG\\{portOut}\K{-}\\{max};{}$\2\6
\&{else}\1\5
${}\\{trans\_s}\MG\\{portOut}\K\\{speed}-\\{rotation};{}$\2\6
\&{if} ${}((\\{speed}+\\{rotation})\G\\{max}){}$\1\5
${}\\{trans\_s}\MG\\{starboardOut}\K\\{max};{}$\2\6
\&{else} \&{if} ${}((\\{speed}+\\{rotation})\Z{-}\\{max}){}$\1\5
${}\\{trans\_s}\MG\\{starboardOut}\K{-}\\{max};{}$\2\6
\&{else}\1\5
${}\\{trans\_s}\MG\\{starboardOut}\K\\{speed}+\\{rotation};{}$\2\6
$\}$ \&{void} \\{setPwm}(\&{transStruct} ${}{*}\\{trans\_s}){}$\1\1\2\2\6
${}\{{}$\1\6
\&{if} ${}(\\{trans\_s}\MG\\{portOut}\G\T{0}){}$\5
${}\{{}$\1\6
${}\.{OCR0A}\K(\\{uint8\_t})\\{trans\_s}\MG\\{portOut};{}$\6
\\{portDirection}(\.{FORWARD});\6
\4${}\}{}$\2\6
\&{else}\5
${}\{{}$\1\6
${}\.{OCR0A}\K(\\{uint8\_t})-\\{trans\_s}\MG\\{portOut};{}$\6
\\{portDirection}(\.{REVERSE});\6
\4${}\}{}$\2\6
\&{if} ${}(\\{trans\_s}\MG\\{starboardOut}\G\T{0}){}$\5
${}\{{}$\1\6
${}\.{OCR0B}\K(\\{uint8\_t})\\{trans\_s}\MG\\{starboardOut};{}$\6
\\{starboardDirection}(\.{FORWARD});\6
\4${}\}{}$\2\6
\&{else}\5
${}\{{}$\1\6
${}\.{OCR0B}\K(\\{uint8\_t})-\\{trans\_s}\MG\\{starboardOut};{}$\6
\\{starboardDirection}(\.{REVERSE});\6
\4${}\}{}$\2\6
\&{if} ${}(\\{trans\_s}\MG\\{portOut}\V\\{trans\_s}\MG\\{starboardOut}){}$\1\5
\\{relayCntl}(\.{CLOSED});\2\6
\&{else}\1\5
\\{relayCntl}(\.{OPEN});\2\6
\4${}\}{}$\2\par
\fi

\inx
\fin
\con
