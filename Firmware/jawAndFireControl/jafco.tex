\input cwebmac

% jafco

\nocon % omit table of contents
\datethis % print date on listing


\N{1}{1}Introduction. This is the firmware portion of Jaw and Fire control.

This will facilitate two actions: opening the jaw to release the floating
object and light the target on fire.

The jaw will close by return-spring so the action will to open it.

Fire is a  sequence of opening the jaw, releasing the butane and firing the
ignitor.

place-holder code below
==========================

Extensive use was made of the datasheet, Atmel ``Atmel ATtiny25, ATtiny45,
ATtiny85 Datasheet'' Rev. 2586Q–AVR–08/2013 (Tue 06 Aug 2013 03:19:12 PM
EDT)
and ``AVR130: Setup and Use the AVR Timers'' Rev. 2505A–AVR–02/02.
\Y\B\X4:Include\X\6
\X5:Types\X\6
\X6:Prototypes\X\6
\X7:Global variables\X\par
\fi

\M{2}\PB{\.{"F\_CPU"}} is used to convey the Trinket clock rate.
\Y\B\4\D$\.{F\_CPU}$ \5
\T{8000000\$U\$L}\par
\fi

\M{3}Here are some Boolean definitions that are used.
\Y\B\4\D$\.{ON}$ \5
\T{1}\par
\B\4\D$\.{OFF}$ \5
\T{0}\par
\B\4\D$\.{OPEN}$ \5
\T{1}\par
\B\4\D$\.{CLOSE}$ \5
\T{0}\par
\B\4\D$\.{SET}$ \5
\T{1}\par
\B\4\D$\.{CLEAR}$ \5
\T{0}\par
\fi

\M{4}\B\X4:Include\X${}\E{}$\6
\8\#\&{include} \.{<avr/io.h>}\SHC{ need some port access }\6
\8\#\&{include} \.{<util/delay.h>}\SHC{ need to delay }\6
\8\#\&{include} \.{<avr/interrupt.h>}\SHC{ have need of an interrupt }\6
\8\#\&{include} \.{<avr/sleep.h>}\SHC{ have need of sleep }\6
\8\#\&{include} \.{<stdlib.h>}\6
\8\#\&{include} \.{<stdint.h>}\par
\U1.\fi

\M{5}Here is a structure to keep track of the state of things.

\Y\B\4\X5:Types\X${}\E{}$\6
\&{typedef} \&{struct} ${}\{{}$\1\6
\\{uint8\_t}\\{count};\2\6
${}\}{}$ \&{statestruct};\par
\U1.\fi

\M{6}\B\X6:Prototypes\X${}\E{}$\6
\&{void} \\{jawcntl}(\\{uint8\_t}\\{state});\SHC{ Jaw open and close }\6
\&{void} \\{fuelcntl}(\\{uint8\_t}\\{state});\SHC{ Fuel on and off }\6
\&{void} \\{igncntl}(\\{uint8\_t}\\{state});\SHC{ on and off }\6
\&{void} \\{fireseq}(\&{statestruct} ${}{*});{}$\6
\&{void} \\{releaseseq}(\&{statestruct} ${}{*}){}$;\par
\U1.\fi

\M{7}
My lone global variable is a function pointer.
This lets me pass arguments to the actual interrupt handlers.
This pointer gets the appropriate function attached by one of the \PB{%
\.{"ISR()"}}
functions.

\Y\B\4\X7:Global variables\X${}\E{}$\6
$\&{void}({*}\\{handleirq}){}$(\&{statestruct} ${}{*})\K\NULL{}$;\par
\U1.\fi

\M{8}
Here is \PB{\\{main}(\,)}.
\Y\B\&{int} \\{main}(\&{void})\1\1 $\{{}$\Y\par
\fi

\M{9}The prescaler is set to clk/16484 by \PB{$\X35:Initialize the timer\X$}.
With \PB{\.{"F\_CPU"}} at 8~MHz, the math goes: $\lfloor{0.5 seconds \times (8 %
\times 10^6 \over 16384}\rfloor = 244$.
Then, the remainder is $256-244 = 12$, thus leaving 244 counts or about 500 ms
until time-out, unless it's reset.

\Y\B\&{statestruct} \\{s\_state}; \X31:Initialize pin inputs\X\X30:Initialize
pin outputs\X\par
\fi

\M{10}
\Y\B\\{jawcntl}(\.{CLOSE});\C{ PB0 }\6
\\{fuelcntl}(\.{OFF});\C{ PB1 }\6
\\{igncntl}(\.{OFF});\C{ PB2 }\par
\fi

\M{11}
Here the timer is setup.
\Y\B\X35:Initialize the timer\X\par
\fi

\M{12}
Of course, any interrupt function requires that bit ``Global Interrupt Enable''
is set; usually done through calling sei().
\Y\B\\{sei}(\,);\par
\fi

\M{13}
Rather than burning loops, waiting 16~ms for something to happen,
the ``sleep'' mode is used.
The specific type of sleep is `idle'.
In idle, execution stops but timers continue.
Interrupts are used to wake it.
\Y\B\X36:Configure to wake upon interrupt\X\par
\fi

\M{14}
This is the loop that does the work.
It should spend most of its time in \PB{\\{sleep\_mode}},
comming out at each interrupt event.

\Y\B\&{for} ( ;  ; \,) $\{{}$\Y\par
\fi

\M{15}
Now we wait in ``idle'' for any interrupt event.
\Y\B\\{sleep\_mode}(\,);\par
\fi

\M{16}
If execution arrives here, some interrupt has been detected.
\Y\B\&{if} ${}(\\{handleirq}\I\NULL{}$)\SHC{ not sure why it would be, but to
be safe }\6
${}\{{}$\1\7
${}\\{handleirq}({\AND}\\{s\_state}){}$;\SHC{ process the irq through it's
function }\6
${}\\{handleirq}\K\NULL{}$;\SHC{ reset so that the action cannot be repeated }\6
\4${}\}{}$\SHC{ end if handleirq }\2\7
$\}{}$\SHC{ end for }\7
\&{return} \T{0};\SHC{ it's the right thing to do! }\7
$\}{}$\SHC{ end main() }\par
\fi

\N{1}{17}Interrupt Handling.

\Y\B\&{void} \\{releaseseq}(\&{statestruct} ${}{*}\\{s\_now}){}$\1\1 $\{{}$\par
\fi

\M{18}
This sequence will proceed only while the button is held.
\Y\B\&{while} ${}(\R(\.{PORTB}\AND(\T{1}\LL\.{PORTB3}))){}$\5
${}\{\,\}{}$\6
$\}{}$\par
\fi

\M{19}


\Y\B\&{void} \\{fireseq}(\&{statestruct} ${}{*}\\{s\_now}){}$\1\1 $\{$ \\{uint8%
\_t}\\{firingstate};\7
\&{enum} \&{firingstates} ${}\{{}$\1\6
${}\\{ready},\39\\{opened},\39\\{warned},\39\\{precharged},\39\\{igniting},\39%
\\{burning}{}$\2\6
${}\};{}$\7
${}\\{firingstate}\K\\{ready}{}$;\par
\fi

\M{20}
This sequence will proceed only while the button is held.
It can terminate after and state.
\Y\B\&{while} ${}(\R(\.{PORTB}\AND(\T{1}\LL\.{PORTB4})))$ $\{{}$\par
\fi

\M{21}
Jaw opens for fire but partly as a warning.
\Y\B\&{if} ${}(\\{firingstate}\E\\{ready}){}$\5
${}\{{}$\1\6
${}\\{firingstate}\K\\{opened};{}$\6
\&{continue};\6
\4${}\}{}$\2\par
\fi

\M{22}
Three 250 ms warning blasts from ignitor and then a 1000 ms delay.
\Y\B\&{if} ${}(\\{firingstate}\E\\{opened}){}$\5
${}\{{}$\1\6
\\{\_delay\_ms}(\T{250});\6
\\{\_delay\_ms}(\T{250});\6
\\{\_delay\_ms}(\T{250});\6
\\{\_delay\_ms}(\T{1000});\C{ human duck time }\6
${}\\{firingstate}\K\\{warned};{}$\6
\&{continue};\6
\4${}\}{}$\2\par
\fi

\M{23}
Fuel opens for precharge, then 250 ms of delay.
\Y\B\&{if} ${}(\\{firingstate}\E\\{warned}){}$\5
${}\{{}$\1\6
\\{\_delay\_ms}(\T{250});\6
${}\\{firingstate}\K\\{precharged};{}$\6
\&{continue};\6
\4${}\}{}$\2\par
\fi

\M{24}
Ignitor on, delay for 250 ms.
\Y\B\&{if} ${}(\\{firingstate}\E\\{precharged}){}$\5
${}\{{}$\1\6
\\{\_delay\_ms}(\T{250});\6
${}\\{firingstate}\K\\{igniting};{}$\6
\&{continue};\6
\4${}\}{}$\2\par
\fi

\M{25}
Ignitor off, and we should have fire now.
\Y\B\&{if} ${}(\\{firingstate}\E\\{igniting}){}$\5
${}\{{}$\1\6
${}\\{firingstate}\K\\{burning};{}$\6
\&{continue};\6
\4${}\}{}$\2\6
$\}{}$\par
\fi

\M{26}
Now set fuel and ignitor off and close jaw.
\Y\B$\}{}$\par
\fi

\M{27}The ISRs are pretty skimpy as they are only used to point \PB{%
\\{handleirq}(\,)} to
the correct function.
The need for global variables is minimized.


This is the vector for the main timer.
\Y\B\C{ Timer ISR }\6
\.{ISR}(\\{TIMER1\_OVF\_vect})\1\1\2\2\6
${}\{{}$\1\6
${}\\{handleirq}\K\NULL;{}$\6
\4${}\}{}$\2\par
\fi

\M{28}
This vector responds to the jaw input at pin PB3 or fire input at PB4.
\Y\B\.{ISR}(\\{PCINT0\_vect})\1\1\2\2\6
${}\{{}$\1\6
\\{\_delay\_ms}(\T{100});\C{ relay settle delay }\6
\&{if} ${}(\R(\.{PORTB}\AND(\T{1}\LL\.{PORTB3}))){}$\1\5
${}\\{handleirq}\K{\AND}\\{releaseseq};{}$\2\6
\&{else} \&{if} ${}(\R(\.{PORTB}\AND(\T{1}\LL\.{PORTB4}))){}$\1\5
${}\\{handleirq}\K{\AND}\\{fireseq};{}$\2\6
\4${}\}{}$\2\par
\fi

\N{1}{29}These are the supporting routines, procedures and configuration
blocks.


Here is the block that sets-up the digital I/O pins.
\fi

\M{30}\B\X30:Initialize pin outputs\X${}\E{}$\6
${}\{{}$\C{ set the jaw port direction }\1\6
${}\.{DDRB}\MRL{{\OR}{\K}}(\T{1}\LL\.{DDB0}){}$;\C{ set the fuel port direction
}\6
${}\.{DDRB}\MRL{{\OR}{\K}}(\T{1}\LL\.{DDB1}){}$;\C{ set the ignition port
direction }\6
${}\.{DDRB}\MRL{{\OR}{\K}}(\T{1}\LL\.{DDB2});{}$\6
\4${}\}{}$\2\par
\U9.\fi

\M{31}\B\X31:Initialize pin inputs\X${}\E{}$\6
${}\{{}$\C{ set the jaw input pull-up }\1\6
${}\.{PORTB}\MRL{{\OR}{\K}}(\T{1}\LL\.{PORTB3}){}$;\C{ set the fire input
pull-up }\6
${}\.{PORTB}\MRL{{\OR}{\K}}(\T{1}\LL\.{PORTB4}){}$;\C{ enable  change interrupt
for jaw input }\6
${}\.{PCMSK}\MRL{{\OR}{\K}}(\T{1}\LL\.{PCINT3}){}$;\C{ enable  change interrupt
for fire input }\6
${}\.{PCMSK}\MRL{{\OR}{\K}}(\T{1}\LL\.{PCINT4}){}$;\C{ General interrupt Mask
register for clear-button}\6
${}\.{GIMSK}\MRL{{\OR}{\K}}(\T{1}\LL\.{PCIE});{}$\6
\4${}\}{}$\2\par
\U9.\fi

\M{32}
Here is a simple function to operate the jaw.
\Y\B\&{void} \\{jawcntl}(\\{uint8\_t}\\{state})\1\1\2\2\6
${}\{{}$\1\6
${}\.{PORTB}\K\\{state}\?\.{PORTB}\OR(\T{1}\LL\.{PORTB0}):\.{PORTB}\AND\CM(%
\T{1}\LL\.{PORTB0});{}$\6
\4${}\}{}$\2\par
\fi

\M{33}
Here is a simple function to operate the fuel.
\Y\B\&{void} \\{fuelcntl}(\\{uint8\_t}\\{state})\1\1\2\2\6
${}\{{}$\1\6
${}\.{PORTB}\K\\{state}\?\.{PORTB}\OR(\T{1}\LL\.{PORTB1}):\.{PORTB}\AND\CM(%
\T{1}\LL\.{PORTB1});{}$\6
\4${}\}{}$\2\par
\fi

\M{34}
Here is a simple function to operate the ignition.
\Y\B\&{void} \\{igncntl}(\\{uint8\_t}\\{state})\1\1\2\2\6
${}\{{}$\1\6
${}\.{PORTB}\K\\{state}\?\.{PORTB}\OR(\T{1}\LL\.{PORTB2}):\.{PORTB}\AND\CM(%
\T{1}\LL\.{PORTB2});{}$\6
\4${}\}{}$\2\par
\fi

\M{35}
A very long prescale of 16384 counts is set by setting certain bits in \PB{%
\.{TCCR1}}.

\Y\B\4\X35:Initialize the timer\X${}\E{}$\6
${}\{{}$\1\6
${}\.{TCCR1}\K((\T{1}\LL\.{CS10})\OR(\T{1}\LL\.{CS11})\OR(\T{1}\LL\.{CS12})\OR(%
\T{1}\LL\.{CS13})){}$;\SHC{Prescale }\6
${}\.{TIMSK}\MRL{{\OR}{\K}}(\T{1}\LL\.{TOIE1}){}$;\C{ Timer 1 f\_overflow
interrupt enable }\6
\4${}\}{}$\2\par
\Q9.
\U11.\fi

\M{36}
Setting these bits configure sleep\_mode() to go to ``idle''.
Idle allows the counters and comparator to continue during sleep.

\Y\B\4\X36:Configure to wake upon interrupt\X${}\E{}$\6
${}\{{}$\1\6
${}\.{MCUCR}\MRL{\AND{\K}}\CM(\T{1}\LL\.{SM1});{}$\6
${}\.{MCUCR}\MRL{\AND{\K}}\CM(\T{1}\LL\.{SM0});{}$\6
\4${}\}{}$\2\par

\U13.\fi


\inx
\fin
\con
