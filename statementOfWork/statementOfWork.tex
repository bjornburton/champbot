
% ************ experimental **************
\PassOptionsToPackage{unicode}{hyperref}
\PassOptionsToPackage{naturalnames}{hyperref}


\documentclass[]{article}

%opening
\title{Champbot Entry 2015}
\author{Bj{\"o}rn, Quentin \& Melinda\\Georgia VT}


% don't get so uptight
\setlength{\hfuzz}{3pc}
\setlength{\vfuzz}{2pc}

%setcounter{tocdepth}{2}

% text and typefaces
\usepackage{listings}
\usepackage{verbatim} %for block comments mostly

\usepackage{fmtcount} % equivalent to \usepackage[super]{nth}

% Euler for math | Palatino for rm | Helvetica for ss | Courier for tt
\usepackage[utf8]{inputenc} % input charater set is the good one
%\usepackage{lcmtt} % for my \texttt
%\renewcommand*\ttdefault{lcmtt}



  \usepackage{tgbonum} % normal typeface
  %\usepackage{urw-garamond}
  %\usepackage[garamond]{mathdesign}
  %\usepackage{kerkis} % normal typeface
  %\usepackage[euler-digits,euler-hat-accent]{eulervm}

%usepackage{makeidx}
\usepackage{textcomp} %for special symbols like degrees, registered & copyright that look good
\usepackage[T1]{fontenc}
\normalfont

\usepackage{amsmath}
\usepackage{amsfonts}
\usepackage{amssymb}
\usepackage{scalefnt}


 \usepackage[pdftex,
 pdfauthor={Bjorn \& Quentin},
 pdftitle={Statement of Work},
 pdfsubject={Champbot},
 pdfkeywords={USV, Boat, Submarine, Remote Control},
 pdfproducer={LaTeX2e with hyperref},
 pdfcreator={pdfLaTeX}]{hyperref}


%\usepackage{tikz} % drawing package. Must be after "\def\pgfsysdriver{pgfsys-tex4ht.def}"
%\usetikzlibrary{matrix,arrows}


% better tables
\usepackage{booktabs}


\usepackage{graphicx}

\usepackage{array}


% ToC tuning
%usepackage{tocloft}% http://ctan.org/pkg/tocloft
%setlength{\cftsecnumwidth}{4 em}% Set length of number width in ToC for \section
%setlength{\cftsubsecnumwidth}{5 em}% Set length of number width in ToC for \subsection
%\setlength{\cftsubsubsecnumwidth}{6em}% Set length of number width in ToC for \subsubsection

% Bib tuning
%makeatletter
%renewcommand\@biblabel[1]{}
%makeatother


\usepackage{url} %sweet!
\usepackage[defaultlines=3,all=true]{nowidow}

%\usepackage[toc]{glossaries}

%\usepackage[title,titletoc,toc]{appendix}
%usepackage[titletoc]{appendix}


% ******** potentialy useful **********
%\usepackage[all]{hypcap}
%microtype makes justification look better, especially with narrow columns. Requires pdflatex. Users of normal latex can get a subset of microtypes features with: pdflatex -output-format=dvi
%booktabs much nicer rules and spacing in tables.
%amsmath makes a lot of common math constructs prettier and easier. Check out the good documentation.
%natbib flexible referencing system.
%subfigure allows you to create sub-figures optionally labeled with subcaptions preceded by (a), (b), etc. Use the [tight] option for better spacing. You can \ref and \subref labels within subfigures to get links to figure 1a) and a). As instructed by the subfigure documentation I now use subfig instead. Having a recent version of the caption package (which subfig includes) is recommended.
%url better than putting URLs inside \texttt{...}. Line-wrapping works better and you don't have to escape tildes.
%textpos allows absolute positioning on a page. Can be useful when press-ganging LaTeX into doing slides or a poster. I sometimes use this to put a DRAFT notice, or publication details in the top margin of a paper.
%\usepackage{booktabs}



\includeonly{}


\newcommand{\nl}{\newline}
\def\degc{~\textdegree C}
\def\degf{~\textdegree F}


% \makeindex

% \makeglossaries

% plain TeX level change for blank pages

 %********* NOT FOR EPUB **********
 \makeatletter % catcode shift of the at symbol from 12 to 11
 \def\cleardoublepage{\clearpage\if@twoside%
 \ifodd\c@page\else
 \vspace*{\fill}
 \hfill
 \begin{center}
 This page intentionally left blank.
 \end{center}
 \vspace{\fill}
 \thispagestyle{empty}
 \newpage
 \if@twocolumn\hbox{}\newpage\fi\fi\fi
 }
 \makeatother % catcode shift of the at symbol from 11 back to 12
 %********* NOT FOR EPUB **********

\begin{document}
\maketitle

\section{Project Summary}
This project will include the design and construction an ROV, closely resembling Champ.
This will successfully navigate the course of the Champbot Challenge as well as execute all optional maneuvers.
The education of ourselves and others will be a large part of this exercise.  

The project will begin with a feature-matrix and a dFMEA (so we don't find ourselves DNF).
Since fire and blades are involved, a risk-assessment will be completed during development.

\section{Objectives}
\subsection{Navigation}
Navigation is non-optional and so has priority and will be implemented and tested first.
In negotiating the slalom around the five buoys, in the prescribed pattern, it must be reasonably quick.
Executing this in a reasonable time will better hold the attention of the spectators.

There will be two drive units providing speed and agility through pirouette turning.


\subsection{Ignition}
The target will be ignited using a gas flame from its mouth.
After all, if Champ will be spewing fire, this is where it should emanate.

It's hoped that the flames will be quite visible to the crowed on the shore.
The fuel will likely be butane, owing to the low pressure, bright flame and relative safety.

Since this is a wet environment, with submersion, ignition will be electro-thermal rather than arc based.

\subsection{Accuracy}
Like some of last year's entries, the float should be dropped from Champ's mouth.
Also like last-years entries, champs neck will be extended to better center into the target.
The mechanics will be a dropping jaw. This must be executed before ignition so that the mouth is clear. 


\subsection{Submersion}
The Submersion for well over 10 seconds will be be supported.
The dive and rise should be startlingly quick, and repeatable, to impress the crowd.

Since this is a short-term dive, with no forward motion planned, power, rather than ballast, will be used. Downward-force is through the temporary re-purposing of the main-drive assembly as a rotary dive-plane. The pitch control will probably be through thrust reaction and not require another servo.

\subsection{Spectacle, Technical and Aesthetics}
Since these all have a point-value equal to the more direct tasks, they will figure heavily into the entire project.


\section{Statement of Work}
There is much to do and many unknowns but the success of this project, in its entirely, is very likely.

The schedule is ordered so that unforeseen delays are unlikely to result in a no-show or disqualification.

\subsection{Schedule}
\begin{tabular}{ r p{.7\textwidth} }
 \\ Week 23 & Main drive/dive units complete; sealed and with mounts.
 \\ Week 26 & Remote control and power systems complete. Firmware functional.
 \\ Week 29 & Buoyant structure complete, along with motor control
 \\ Week 30 & Early lake test, finalize firmware and spring-rates 
 \\ Week 32 & Jaw control added and tested
 \\ Week 34 & Fire-control hardware and firmware added and tested
 \\ Week 36 & Skeleton \& Skin complete and installed
 \\ Week 37 & Final lake test to finalize buoyancy and address and handling issues
 \\ Week 39 & Corrections will be ongoing until the event
\end{tabular}

\subsection{Budget}
Money is always a very limited resource.
Like many of last years entries, rather than the obvious industrial or hobby parts this project will reuse and re-purpose household appliances and material.
Simplicity should also keep costs down.

At this time it appears that the cost could be well under \$500.

   
\subsection{Team Composition and Expertise}
Bj{\"o}rn is an autodidactic engineer and life-long maker, building his first remote-control device, an optically controlled Automaton, in 1973.  
Quentin is a brilliant \ordinalnum{1} grade student at Georgia Elementary \& Middle School (GEM), and Bj{\"o}rn's son, who loves electronics and chemistry.
Melinda is an industrial designer.


\section{Facilities}
All work will be performed within the home.
Github will be use for both source control and documentation at this location:
\url{https://github.com/bjornburton/champbot}


\end{document}